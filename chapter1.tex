\newpage
\chapter{Введение}
 В 1897 году Дедекинд сформулировал задачу нахождения числа элементов в
 свободных дистрибутивных решетках или, что эквивалентно,
 нахождения числа $\psi(n)$ всех монотонных булевых функций (МБФ) 
 от $n$ переменных. С тех пор ученые исследую это проблему в двух
 направленияx.
 Первое - это вычисление этого числа для фиксированного $n$
 выведением соответствующих формул для него или специальными
 алгоритмами для подсчета.
 Второе направление (первое является не очень успешным) - это 
 оценка и приближенные подсчеты этого числа - много формул для
 вычисления $\psi(n)$ получены в \cite{Kisielewicz, Kleitman, Korshunov81, Korshunov02}.
 Несмотря на их усилия, точные значения $\psi(n)$ известны
 только для $n \leq 8$ \cite{Fidytek, Tombak, Wiedemann}: \\
\begin{center}
 \begin{tabular}{ r | r | r } 
  $n$ & $\psi(n)$ & Кем вычислено \\
  \hline
  0 & 	2				& \T Р. Дедекинд, 1897 \\
  1 & 	3 				& Р. Дедекинд, 1897 \\
  2 & 	6 				& Р. Дедекинд, 1897 \\
  3 & 	20 				& Р. Дедекинд, 1897 \\
  4 & 	168 				& Р. Дедекинд, 1897 \\
  5 & 	7 581 				& Р. Черч, 1940 \\
  6 &	7 828 354			& М. Уорд, 1946 \\
  7 &	2 414 682 040 998 		& Р. Черч, 1965 \\
  8 &	56 130 437 228 687 557 907 788 	& \B Д. Видеман, 1991 \\
  \hline
  \end{tabular}
\captionof{table}{ \label{Table-psi(n)}
  $\psi(n)$ для $0 \leq n \leq 8$ и история вычисления.
  }
\end{center}

Чтобы почуствовать всю сложность проблемы отметим, что в 1991 году
Видеман использовал процессор Cray-2 и вычисление $\psi(8)$
заняло 200 часов. Почти целый век потребовалось, чтобы вычислить
последние 4 значения $\psi(n)$ из Таблицы \ref{Table-psi(n)}.\par
В этой работе мы предлагаем новый алгоритм, использующий 
динамическое программирование,
для генерирования (и подсчета) всех МБФ.
