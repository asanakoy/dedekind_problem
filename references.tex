\addcontentsline{toc}{chapter}{Литература}


\begin{thebibliography}{9}
\bibitem{bakoev2012} 
V. Bakoev, Thirteenth International Workshop on Algebraic and Combinatorial Coding Theory
June 15-21, 2012, Pomorie, Bulgaria, pp. 47--52
\bibitem{bakoev2003}
V. Bakoev, Generating and identification 
	of monotone Boolean functions, Mathematics 
	and education in mathematics, 
	Sofia (2003), pp. 226--232.
	
\bibitem{bakoev2002} V. Bakoev, Some properties of one matrix structure at monotone Boolean func-
tions, Proc. EWM Intern. Workshop Groups and Graphs, Varna, Bulgaria (2002),
pp. 5--8.
\bibitem{dezert} J. Dezert, F. Smarandache, On the generation of hyper-powersets for the DSmT,
Proc. of the 6th Int. Conf. of Information Fusion (2003), pp. 1118--1125.
\bibitem{Fidytek} R. Fidytek, A. Mostowski, R. Somla, A. Szepietowski, Algorithms counting mono-
tone Boolean functions, Inform. Proc. Letters 79 (2001), pp. 203-209.
\bibitem{Kisielewicz} A. Kisielewicz, A solution of Dedekinds problem on the number of isotone Boolean
functions, J. reine angew. math., Vol. 386 (1988) pp. 139--144.
\bibitem{Kleitman} D. Kleitman, On Dedekinds problem: the number of monotone Boolean functions,
Proc. of AMS, 21(3) (1969), pp. 677-682.
\bibitem{Korshunov81} A. Korshunov, On the number of monotone Boolean functions, Problemy Kiber-
netiki, Vol. 38, Moscow, Nauka (1981), pp. 5-108 (in Russian).
\bibitem{Korshunov02} A. Korshunov, I. Shmulevich, On the distribution of the number of monotone
Boolean functions relative to the number of lower units, Discrete Mathematics,
257 (2002), pp. 463-479.
\bibitem{kmath094} http://mathpages.com/home/kmath094.htm, Generating the M. B. Functions
\bibitem{ronzeno} http://angelfire.com/blog/ronz/, Ron Zeno's site
\bibitem{Tombak} M. Tombak, A. Isotamm, T. Tamme, On logical method for counting Dedekind
numbers, Lect. Notes Comp. Sci., 2138, Springer-Verlag (2001), pp. 424--427.
\bibitem{Wiedemann} D. Wiedemann, A computation of the eighth Dedekind number, Order, no. 8
(1991), pp. 5--6.
\bibitem{MDNF} Минимизация логических функций методом Куайна [Электрон. ресурс]. -- 
  \underline{http://tinyurl.com/Minimazing-DNF}
\bibitem{Moschenski-2001} А. В. Мощенский, В. А. Мощенский, Курс математичекой логики, Минск БГУ (2001),
стр. 30.
\bibitem{Kronecker_product} Произведение Кронекера [Электрон. ресурс]. -- 
  \underline{http://tinyurl.com/Kronecker-product}
\end{thebibliography}

