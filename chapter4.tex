\chapter{Заключение}
Результаты в Таблице \ref{Table_cases_count_for_Mn} кажутся оптимистичными, 
особенно если мы сравним их со значениями $\psi(n)$, данными в
Таблице \ref{Table-psi(n)}. Главная и всё ещё открытая прблема заключается в том,
чтобы разработать эффективный способ выисления в Случае 3. Также требуется решить
несколько второстепенных задач -- например, представление и суммирование длинных целых
чисел, эффективное использование памяти, особенно дял матриц $M_n$ и $Res_n$, и т.д.
Их эффективное решение существенно сократит время требуемое для вычисления $\psi(7)$ и 
$\psi(8)$ и, возможно, позволит нам вычислить $\psi(9)$ за приемлемое время.

\begin{center}
 \begin{tabular}{ c | r | r | r | r | r}
  $n$ & $(4^n - 2 \cdot 3^n + 2^n)/2$ & В случае 1 & В случае 2 & В случае 3 & Копии\\
  \hline
  6 & 	1351	& 211	& 26	& 544	& 	570	\T  \\
  7 & 	6069	& 665	& 57	& 2645	& 	2702	\\
  8 & 	26335	& 2059	& 120	& 12018	& \B	12138	\\
  \hline
  \end{tabular}
\captionof{table}{\label{Table_cases_count_for_Mn}
 Число элементов для каждого из случаев в матрице $Res_n$
  }
\end{center}